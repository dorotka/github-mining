% !TeX root = ../main.tex
% Add the above to each chapter to make compiling the PDF easier in some editors.

\chapter{Introduction}\label{chapter:introduction}
\textit{“Design and programming are human activities; forget that and all is lost”} – Bjarne Stroustrup

\vspace{5mm}

Software development is an inherently team-based activity. Empirical analysis can be used to examine development processes and programmers’ behaviors. Consequently, it can yield important insights for the development community. Such knowledge can enable software engineering teams to make decisions that will benefit the team and the product. The key to building a successful product is to understand how developers work and what makes them thrive or struggle. Many characteristics have an impact on developer’s quality. In this paper, I analyze one of those characteristics to answer the question of whether \textit{good developers} are more focused. 
\par
Assessing developers’ quality is not easy. There are many criteria to be taken into account, and some of them may be subjective. In this paper, an automated process was used to select \textit{good developers}. \textit{Good developers} were chosen as those who introduce 0\% of bugs. It was a harsh criterion based only on the bug-introducing lines and it was not optimal; developers selected as \textit{good} in this paper were not ultimately superior, as well as those selected as \textit{bad} were not ultimately inferior. The terms had to be chosen for the sake of the distinction and ease of describing further research. To analyze focus of \textit{good developers}, authors had to be first divided into two groups— \textit{good} and \textit{bad}. The details of the method used are described in the Methods section followed by explanations of its shortcomings in the Threats to Validity section.

\section{Overview of Contents}
Following the introduction, this paper contains eight main parts. The Background section introduces the research questions and Prior Work discusses related research. The Methods section describes the data gathering and analysis process; it is divided into five subsections. The Dataset and Repository Selection sections talk about the data sources and the project selection criteria. The Good Developers Selection part explains the process of identifying the fix-inducing changes and developer classification criteria. In the Focus Analysis subsection, you will find the focus definition and two ways the focus was measured in this paper. Lastly, the Testing subsection describes the testing and verification done on the scripts and the resulting data. The Threats to Validity section discusses the precision of the measurements used and the known limitations of the research. Results, and Implications for Software Practice sections include a description and a significance of the findings. The Future Research section discusses multiple ways the topic of focus and developers’ quality can be studied further. Finally, the Conclusions contain the summary of all the ideas and findings.


