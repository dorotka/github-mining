% !TeX root = ../main.tex
% Add the above to each chapter to make compiling the PDF easier in some editors.

\chapter{Future Research}\label{chapter:future}

The findings in this research showed clearly that \textit{good developers }tended to focus over a shorter amount of time but, overall, they diversified the technologies they worked with. Additionally, \textit{good developers} were more focused on all the popular extensions except for .rb, where the focus level was similar for all programmers. These findings provide details that facilitate the understanding of the processes used by developers. My research can be further expanded by performing focus analyses that examine temporal focus switching. In this paper, the developer who worked exclusively on .java extensions was described as focused, while the developer who worked on three .java, two .py, and .config was not focused. The order of focus-switching between the technologies was not taken into account. The next step to evaluate focus level more accurately is to analyze the time and order of the switching. For example, if both developer A and developer B worked on six files (.java, .java, .java, .py, .py, .config), but developer A worked first on three .java files, then on two .py files, and last on .config, while developer B worked on .java, .py, .java, .config, .py, .java in that order, then the focus level of the two developers was different. Developer A seems more focused than developer B.\par

Additionally, a survey method could be employed to investigate the developers’ perception of their focus and the reasons why they focus on different technologies. There are many reasons why a developer would switch her attention between different extensions—it could be dictated by technological dependencies. For example, if a developer worked on an .html file, she would likely also work on a .css or .js file. The switch could also be caused by a resource delay. For example, if a front-end developer needed to wait for a back-end developer to finish an endpoint before she could integrate it into the UI. Then, she would switch to different tasks utilizing different technologies in the meantime. Similarly, a developer could be waiting for the tests or deployment to finish and in the meantime she worked on another part of the project. Finally, certain changes just require modifications within files with multiple extensions. It is important to understand not only how developers behave but also the reasons behind their practices. \par

Finally, there are many other characteristics that could be explored for their relation to developers’ quality. Additional study could be performed to find out whether \textit{good developers} are more consistent. One could suspect that \textit{good developers} write code regularly. Possibly, they follow a schedule that optimizes their focus and mental abilities, like working only for six hours a day. It is an interesting idea to explore developers’ work scheduling further.

